\documentclass{mydoc}

\author{张丘洋}
\title{\dabiaosong ComboTree 设计与实现}

\begin{document}

\maketitle

\section{概览}

根据 ComboTree 的设计,将数据结构分为 \code{CLevel}、\code{BLevel} 和 \code{ALevel}。
其中 \code{BLevel::Entry} 中的键值对缓存使用 \code{KVBuffer},\code{CLevel::Node}
使用 \code{SortBuffer} 来保存数据。

下面介绍的一致性都是系统的崩溃一致性,只能保证在整个系统崩溃后通过重启来恢复,
若某个线程在中途崩溃了则无法修复,因为该机制没有运行时检查。

\section{KVBuffer}

\code{KVBuffer} 使用\textbf{前缀压缩}保存若干键值对数据。其数据结构定义为:

\begin{codes}{C++}
struct KVBuffer {
  union {
    uint16_t meta;
    struct {
      uint16_t prefix_bytes : 4;  // 共同前缀字节数
      uint16_t suffix_bytes : 4;  // 后缀字节数,与前缀之和为8
      uint16_t entries      : 4;  // 目前保存的键值对数
      uint16_t max_entries  : 4;  // 最大可保存的键值对数
    };
  };
  uint8_t buf[112];
};
\end{codes}

\code{KVBuffer} 包含了 2 个字节的元数据 \code{meta},以及 112 个字节的 \code{buf}
来保存键值对。键值对的保存是\textbf{追加}的。

当后缀字节数为 1,一个 \code{KVBuffer} 能够保存 12 个键值对;当后缀字节数为 2
时,能够保存 11 个键值对。

由于使用了前缀压缩,键的大小是不固定的,但是值的大小固定为 8 个字节,为了使值
在保存时能够 8 字节\textbf{对齐},以及使键聚集在一起,让\textbf{所有键都位于同一个
Cache Line 中},\code{buf} 从左到右顺序保存键,从右到左顺序保存值,如下图所示。

\begin{figure}[!htbp]
\centering
\begin{tikzpicture}[x=1em,y=1em,every node/.style={draw,line width=1pt,inner sep=.5em,minimum height=2.2em,minimum width=2.5em}]
  \node at (0,0)                          (0) {key(0)};
  \node[right=-1pt of 0.east,anchor=west] (1) {key(1)};
  \node[right=-1pt of 1.east,anchor=west] (2) {...};
  \node[right=-1pt of 2.east,anchor=west] (3) {key(n)};
  \node[right=-1pt of 3.east,anchor=west] (4) {...};
  \node[right=-1pt of 4.east,anchor=west] (5) {value(n)};
  \node[right=-1pt of 5.east,anchor=west] (6) {...};
  \node[right=-1pt of 6.east,anchor=west] (7) {value(1)};
  \node[right=-1pt of 7.east,anchor=west] (8) {value(0)};
\end{tikzpicture}
\end{figure}

\subsection{崩溃一致性}

\subsubsection{插入}

插入操作首先将值和键追加到相应的位置,刷写一次 Cache Line,之后更新
8 字节元数据使其生效。元数据小于 8 个字节所以是原子操作。

\subsubsection{更新}

更新操作只需要更新 8 个字节的值,是原子操作。

\subsubsection{删除}

由于 B 层数据结构比较紧凑,现在没有设置删除标记位。

目前删除操作的执行步骤为:

\begin{enumerate}
  \item 查找该键对应的位置
  \item 若键处于最后一个位置,则直接原子性更新元数据即可。若不是,执行下一步
  \item 将处于最后一个位置的键拷贝到当前位置。若该步骤完成后系统崩溃,则在
        下一次启动时可以通过扫描重复的键判断出此步骤出错
  \item 将处于最后一个位置的值拷贝到当前的位置。此步骤完成后出错与上一步相同
  \item 原子性更新元数据,这样就把最后一个位置的键值对移动到了被删除的地方
\end{enumerate}

\section{SortBuffer}

\code{SortBuffer} 是 C 层每个节点保存数据的数据结构。\code{SortBuffer} 整体上
与 \code{KVBuffer} 类似,不同之处是有额外的 6 个字节来保存排序索引。

\code{SortBuffer} 的定义如下所示,能够看到与 \code{SortBuffer} 相比多了一个
\code{sorted\_index}。

\begin{codes}{C++}
struct SortBuffer {
  union {
    uint64_t header;
    struct {
      union {
        uint16_t meta;
        struct {
          uint16_t prefix_bytes : 4;  // LSB
          uint16_t suffix_bytes : 4;
          uint16_t entries      : 4;
          uint16_t max_entries  : 4;  // MSB
        };
      };
      uint8_t sorted_index[6];
    };
  };
  uint8_t buf[112];
};
\end{codes}

\code{sorted\_index} 虽然是个 \code {uint8\_t} 数组,但是作为 \code{uint4\_t[12]}
使用的,也就是 12 个 4 比特数组。之前说过每个缓存中最多能保存 12 个数据,而 4
个比特可以表示 0--11。这个比特数组中第 n 个数据表示从小到大第 n 个键在 \code{buf}
里的下标。

\code{sorted\_index} 和其他的元数据一起组成了 8 个字节,所以同时修改他们是原子
的。

\code{sorted\_index} 数组除了用于保存排序数据之外,还用于\textbf{垃圾回收},
由于该数据包含了 12 个数据,在保存时,数组从前往后是目前有效数据的排序数组,
从后往前是空闲位置的下标。

\subsection{崩溃一致性}

\subsubsection{插入}

插入数据时,首先从 \code{sorted\_index} 的最后取出一个空闲的位置,将
键值对写入对应的位置中,并刷写一次 Cache Line,之后原子更新 8 个字节
的头部数据即可。

\subsubsection{更新}

更新操作只需要更新 8 个字节的值,是原子操作。

\subsubsection{删除}

删除只需要修改 8 个字节的头部数据即可(修改目前的有效键值数量以及 \code{sorted\_index})。

\section{CLevel}

\begin{codes}{C++}
// sizeof(Node) == 128
struct __attribute__((aligned(64))) Node {
  enum class Type : uint8_t {
    INVALID,
    INDEX,
    LEAF,
  };

  Type type;
  uint8_t __padding[7]; // used to align data
  union {
    // uint48_t pointer
    uint8_t next[6];    // used when type == LEAF. LSB == 1 means NULL
    uint8_t first_child[6]; // used when type == INDEX
  };
  union {
    // contains 2 bytes meta
    KVBuffer leaf_buf;  // used when type == LEAF, value size is 8
    KVBuffer index_buf; // used when type == INDEX, value size is 6
  };
};

// sizeof(CLevel) == 6
class CLevel {
  ......
 private:
  uint8_t root_[6]; // uint48_t pointer, LSB == 1 means NULL
};
\end{codes}

CLevel 采用的是 \textbf{B+ 树},每个节点大小为 \textbf{128 字节,两个 Cache 行},
与 \code{BLevel::Entry} 的大小相同。节点使用 \textbf{\code{SortBuffer}} 来保存
键值对和子节点。节点\textbf{没有保存父指针},通过函数的递归来传递父指针,这样
避免了扩展时父指针的修改,也减少了父指针的存储。

\textbf{C 层使用 \code{SortBuffer} 的原因是}:在扩展时,需要按顺序依次扩展,
若 C 层无序,则扩展时每个 C 层节点都必须重新排序,导致扩展时间较长。另外若采用
\code{KVBfuffer},在 C 层分裂删除旧节点中后半部分数据时会比较费时。

C 层的中间节点在保存子节点\textbf{指针时使用 6 个字节来保存},也就是 \code{KVBuffer}
键值对中的值大小为6,从而有更大的扇出。

无论是子节点指针还是根节点指针都通过将指针的\textbf{最低比特位置 1} 来表示 NULL,
这样避免了多字节的拷贝,并且 \emph{Optimizing Systems for Byte-Addressable NVM
by Reducing Bit Flipping} 论文中提到这种方式可以减少比特翻转从而提高 NVM 寿命。

整个 C 层 \code{KVBuffer} 的\textbf{前缀压缩字节数是一样的},前缀字节数和共同前缀可以由
对应的 \code{BLevel::Entry} 来确定。

\textbf{C 层无锁},由 \code{BLevel::Entry} 对应的锁来提供并发访问控制。

\subsection{崩溃一致性}

\subsubsection{插入}

若插入时不涉及到分裂,则一致性由 \code{SortBuffer} 提供。下面主要叙述节点
分裂的步骤:

\begin{enumerate}
  \item 创建新的节点,将旧节点后半部分数据拷贝到新的节点,更新新节点的 \code{next}
        指针,持久化新节点。此时对旧节点没有任何的修改,若此步骤后崩溃没有任何问题。
        注意,分裂会在插入之后进行,也就是说\textbf{先插入后分裂}而不是先分裂后插入。
  \item 将新的节点插入到父节点中。插入的一致性由 \code{SortBuffer} 保证。若此步骤
        后崩溃,在重启扫描时会发现该问题进行修复。
  \item 修改旧节点的 \code{next} 指针。此步骤后崩溃与上一步骤一样。
  \item 修改旧节点的元数据信息,从而删除后半部分数据。此步骤后崩溃,则下一步
        父节点可能需要分裂但却没有执行,在重启扫描时可以修复该问题。
  \item 函数返回到父节点,父节点判断当前节点是否满,若满则该节点从第 1 步开始
        执行分裂,若不满,则分裂完成。这里父节点需要判断是否分裂是因为
        需要\textbf{先插入后分裂},否则第 2 步执行时若父节点满则会出现问题。
        此步骤后崩溃可能父节点分裂没有执行,重启扫描可以修复。
\end{enumerate}

\subsubsection{更新}

更新一致性由 \code{SortBuffer} 提供。

\subsubsection{删除}

为了实现简单,删除目前\textbf{不会合并节点},只会在子节点的 \code{SortBuffer}
中删除对应的数据,一致性由 \code{SortBuffer} 提供。

\section{BLevel}

B 层的主要数据结构定义如下所示:

\begin{codes}{C++}
// sizeof(Entry) == 128
struct __attribute__((aligned(64))) Entry {
  uint64_t entry_key;     // 8 bytes
  CLevel clevel;          // 6 bytes
  KVBuffer<48+64,8> buf;  // 114 bytes
};

struct BRange {
  uint64_t start_key;
  uint32_t logical_entry_start;
  uint32_t physical_entry_start;
  uint32_t entries;
};

class BLevel {
  ......
 private:
  ......
  Entry* __attribute__((aligned(64))) entries_;
  size_t nr_entries_;
  std::atomic<size_t> size_;

  // 锁粒度为 Entry
  std::shared_mutex* lock_;

  // B 层分为 EXPAND_THREADS 个 range,这里多分配一个 range 减少编程难度
  BRange ranges_[EXPAND_THREADS+1];

  // 每个 range 分为多个小区间,记录每个小区间中的键值对数
  uint64_t interval_size_;
  uint64_t intervals_[EXPAND_THREADS];
  std::atomic<size_t>* size_per_interval_[EXPAND_THREADS];

  // 记录扩展时每个区间已经扩展的 entry 数目以及最大的键
  static std::atomic<uint64_t> expanded_max_key_[EXPAND_THREADS];
  static std::atomic<uint64_t> expanded_entries_[EXPAND_THREADS];
};
\end{codes}

B 层是由 \code{BLevel::Entry} 组成的数组。每个 \code{Entry} 的大小
为 \textbf{128 字节,两个 Cache 行},与 \code{CLevel::Node} 的大小相同。
节点使用 \textbf{\code{KVBuffer}} 来保存键值对和子节点。

在插入数据时,会先插入到 Entry 的 \code{KVBuffer} 缓存中,如果发现缓存满了,
会先把 \code{KVBuffer} 刷入到对应的 C 层再进行插入。

\code{BLevel::Entry} 中的 \code{buf} 和 \code{CLevel::Node} 的 \code{buf}
大小相同,在 B 层 Entry \textbf{第一次将数据刷写到 C 层时,会直接复制 \code{buf}},
而不需要一个一个地将数据插入到 C 层(由于 C 层节点有排序数组所以还需要获得
排序数组)。

\subsection{B 层的大区间 \code{BRange} 和小区间 \code{interval}}

为了能够实现多线程扩展,将 B 层分为多个 \code{BRange},\code{BRange}
内部是连续的,\code{BRange} 之间不一定连续。\code{BRange} 的个数等于
扩展线程的个数,扩展时每个线程扩展一个 \code{BRange}。

为了尽量让每个线程扩展的数据量相同,将每个 \code{BRange} 再细分为多个
\code{interval},\code{interval} 目前最大为 128。每个 \code{interval}
会记录该区间内的键值对数,在开始扩展时,会将所有 \code{BRange} 的
\code{interval} 合在一起,并将其根据键值数尽量等分,分给不同的扩展线程。
也就是说,在扩展时,每个扩展线程对应一个新的 \code{BRange},但并不对应
一个旧的 \code{BRange}。

\subsection{崩溃一致性}

\subsubsection{插入、更新和删除}

由 \code{KVBuffer} 和 C 层提供一致性。

\subsubsection{扩展}

B 层扩展时,会持久化每个 range 中当前已经扩展的最大键值(目前的实现没有持久化),
旧的 B 层只会在扩展完成后才会被删除。

\section{ALevel}

A 层与原论文一致。A 层不需要修改,整个位于内存中,重启时重新生成即可。

\section{NVM 管理}

使用 PMDK 中的 \code{libpmem} 来管理 NVM。

B 层和 C 层保存在 NVM DAX 文件系统的两个文件中,在扩展时会创建新的文件,扩展完成后
删除旧的文件。

\end{document}